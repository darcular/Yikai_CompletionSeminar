

    \usetikzlibrary{
        pgfplots.groupplots,
    }
    \pgfplotsset{
        % so that the new feature of centering the `nodes near coords' text is applied
        compat=1.9,
        %
        % create a style which will be used to show the sum values
        % (borrowed from <http://tex.stackexchange.com/a/162389/95441>)
        show sum on top/.style={
            /pgfplots/scatter/@post marker code/.append code={
                \node [
                    black,
                    rotate=90,
                    anchor=west,
                    at={(normalized axis cs:
                            \pgfkeysvalueof{/data point/x},
                            \pgfkeysvalueof{/data point/y})
                    },
                ] {\pgfmathprintnumber{\pgfkeysvalueof{/data point/y}}};
            },
        },
    }
    \pgfplotstableread{
        Criterion   {Clusters with surrounding abnormal traffic}  {Clusters without surrounding abnormal traffic}     
        3         41      20     
        4         86      21     
        5         396     149     
        6         267     98     
        7         198     40     
        8         149     36
        9         103     25
        10        89      15
        $>$10     380     50     
    }\data
    % get number of columns in the table
    \pgfplotstablegetcolsof{\data}
        \pgfmathtruncatemacro{\NoOfCols}{\pgfmathresult - 1}
    % we also need this value again on a "zero basis" for a style
        \pgfmathtruncatemacro{\NoOfColsZeroBased}{\NoOfCols-1}
    % add column of the sum of the values in each row
    % (because you did the rest also with loops, so here is the loop solution for it)
    % (inspired by <http://tex.stackexchange.com/a/281436/95441>)
    \pgfplotstablecreatecol[
        create col/assign/.code={
                % create and initialize a variable that stores the sum values
                \pgfmathsetmacro\RowSum{0}
            % cycle through the columns
            \pgfplotsforeachungrouped \i in {1,...,\NoOfCols} {
                \pgfmathsetmacro\RowSum{\RowSum+\thisrowno{\i}}
            }
            % set the calculated value to the corresponding cell
            \pgfkeyslet{/pgfplots/table/create col/next content}{\RowSum}
        }
    ]{Sum}{\data}


    \begin{tikzpicture}
        \begin{groupplot}[
            group style={
                columns=2,
                xlabels at=edge bottom,
                ylabels at=edge left,
                horizontal sep=1cm,
                group name=plots,
            },
            legend cell align={left},
            ymin=0,
            enlarge y limits={upper,value=0.1},
            ylabel=Number of Clusters,
            xlabel=Cluster size (members),
            ybar stacked,
            y=0.015cm,
            % instead of making the ticks invisible, just set the length to zero
            major tick length=0pt,
            xtick=data,
            xticklabels from table={\data}{Criterion},
            legend columns=1,
            % -----
            % define the number style for the `nodes near coords' where
            % only values greater than 5 should be shown
            % (adapted from <http://tex.stackexchange.com/a/59961/95441>
            nodes near coords={
                \pgfkeys{/pgf/fpu=true}
                \pgfmathparse{\pgfplotspointmeta-5}
                \pgfmathfloatifflags{\pgfmathresult}{+}{%
                    \pgfmathprintnumber[
                        precision=0,
                    ]{\pgfplotspointmeta}\%
                }{}
                \pgfkeys{/pgf/fpu=false}
            },
%            % because rotating the values would cause them to overlap
%            % I just provide the solution to that in commented form
%            nodes near coords style={
%                rotate=90,
%                font=\footnotesize,
%            },
            % to the last plot the sum value should be added as a node
            % for which we use the created style
            every axis plot no \NoOfColsZeroBased/.append style={
                show sum on top,
            },
        ]
        \nextgroupplot[
            % instead of shifting the `xlabel' just provide a title
            title=,
            legend to name=grouplegend,
        ]
            % add all data columns to the axis
            \foreach \i in {1,...,\NoOfCols} {
                \ifthenelse{\i=1}
                {
                    \addplot+ [
                        % this has to be given because we want to use a different
                        % value than the default `y' value
                        point meta=explicit,
                    ] table [
                        x expr=\coordindex,
                        y index=\i,
                        % here we calculate the number which we want to be plotted
                        meta expr=\thisrowno{\i}/\thisrow{Sum}*100,
                    ] {\data};
                    % extract the corresponding column header name from the table
                    \pgfplotstablegetcolumnnamebyindex{\i}\of{\data}\to{\colname}
                    % use it as legend entry
                    \addlegendentryexpanded{\colname}

                }
                {
                    \addplot+ [
                    % this has to be given because we want to use a different
                    % value than the default `y' value
                    point meta=explicit,
                    ] table [
                        x expr=\coordindex,
                        y index=\i,
                        meta expr=0,
                    ] {\data};
                    % extract the corresponding column header name from the table
                    \pgfplotstablegetcolumnnamebyindex{\i}\of{\data}\to{\colname}
                    % use it as legend entry
                    \addlegendentryexpanded{\colname}

                }
            }
%        % I removed the second plot because there is nothing different here
%        \nextgroupplot
        \end{groupplot}

        % \draw [dashed] (plots c1r1.south west) -- ++(0,-1.5);
        % \draw [dashed] (plots c1r1.south east) -- ++(0,-1.5);

        \node [inner sep=0pt,anchor=north, yshift= 7ex, xshift= -2.2ex]
            at (plots c1r1.north) {\ref{grouplegend}};

    \end{tikzpicture}

